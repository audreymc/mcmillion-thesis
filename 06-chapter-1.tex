\chapter{Introduction} \label{chap:chap-1}

Financial data, and more specifically short-term equity price series, are considered to be a special class of time series given their unique challenges they present in modeling and forecasting. Though equity prices are derived from empirical asset valuation, the underlying valuation theory and the corresponding empirical price series contain a level of uncertainty and stochasticity \cite{TsayRuey}. Because of this, financial time series often likened to random walk processes, which are, by definition, unpredictable \cite{KHOSRAVI2023121012}. However, there are certain observable and well-researched characteristics of financial return\footnote{Generally, in statistics literature, the terms ``financial returns'' is synonymous with the term \textit{continuously compounded returns} and \textit{log returns}, which are defined as $r_t = \log \frac{P_t}{P_{t-1}}= \log P_t - \log P_{t-1}$ where $P_{t}$ is the price of the stock at time $t$. \cite{TsayRuey}} series which include serial dependence without correlation, tail heaviness (excess kurtosis), asymmetry, and conditional heterostochasticity or volatility clustering \cite{brockwell2016introduction}.  These aspects of financial time series cannot be modeled using traditional linear models such as autoregressive integrated moving average (ARIMA) models, which assume that the variance of the series is constant over time, the errors are normally distributed, and the series is stationary \cite{brockwell2016introduction}. Further, financial time series are often subject to concept shifts, where distributional and statistical properties of the series fluctuate over time or in reaction to external events \cite{KHOSRAVI2023121012}.

The same characteristics that complicate modeling and predicting financial time series extend to identifying anomalous observations in these series. An anomalous observation in a series of financial returns may be indicative of strong investor sentiment based on company performance or market-wide events, and many empirical studies attempt to model these extreme observations for profit-seeking purposes based on external variates. Alternatively, an anomaly in these data may be indicative of market manipulation that is reflected in artificial price movements that are not driven by fundamentals or systemic factors, e.g. price ramping or pump-and-dump schemes. In these cases, effective and accurate anomaly detection is integral to maintaining fair and efficient financial markets.  

In statistical inference, an \textit{anomaly} is defined as ``an observation which is suspected of being partially or wholly irrelevant because it is not generated by the stochastic model assumed'' \cite{Anscombe1960RejectionOO}. Standard statistical anomaly detection methods are based on thresholds derived from assumed distributions. Although statistical distributions such as Normal, Log-Normal, Cauchy,  and Laplace have been proposed to model the marginal distributions of financial returns, the common phenomenon of concept drift in financial time series renders these methods inadequate \cite{TsayRuey} \cite{Pokharel_laplace}. Concept drift or distribution shift ...
